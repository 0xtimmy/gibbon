
% This varies by conference, sometimes 9pt:
\documentclass[preprint,10pt]{./bibs/sigplanconf}
% The following \documentclass options may be useful:
%
% 10pt          To set in 10-point type instead of 9-point.
% 11pt          To set in 11-point type instead of 9-point.
% authoryear    To obtain author/year citation style instead of numeric.


%% Common packages we use:

%% This is our standard package for code formatting:
\usepackage{listings}
\usepackage{amsmath}
\usepackage{amsthm}
\usepackage{hyperref}
\usepackage{graphicx}
% \usepackage{mathabx}
% \usepackage{mathpartir}

\usepackage[noabbrev]{cleveref}
\usepackage{enumitem}

\newcommand{\gramdef}{\; ::= \;}
\newcommand{\gramor}{\; | \;}
\newcommand{\keywd}[1]{\; \texttt{#1} \;}

%----------------------------------------
%% \usepackage[colorinlistoftodos,prependcaption,textsize=tiny]{todonotes}
%% \usepackage{xargs}
%% \newcommandx{\unsure}[2][1=]{\todo[linecolor=red,backgroundcolor=red!25,bordercolor=red,#1]{#2}}
%% \newcommandx{\info}[2][1=]{\todo[linecolor=OliveGreen,backgroundcolor=OliveGreen!25,bordercolor=OliveGreen,#1]{#2}}
%% \newcommandx{\change}[2][1=]{\todo[linecolor=blue,backgroundcolor=blue!25,bordercolor=blue,#1]{#2}}
%% \newcommandx{\inconsistent}[2][1=]{\todo[linecolor=blue,backgroundcolor=blue!25,bordercolor=red,#1]{#2}}
%% \newcommandx{\improvement}[2][1=]{\todo[linecolor=Plum,backgroundcolor=Plum!25,bordercolor=Plum,#1]{#2}}
%% \newcommandx{\resolved}[2][1=]{\todo[linecolor=OliveGreen,backgroundcolor=OliveGreen!25,bordercolor=OliveGreen,#1]{#2}} % use this to mark a resolved question
%% \newcommandx{\thiswillnotshow}[2][1=]{\todo[disable,#1]{#2}} % will replace \resolved in the final document
% ----------------------------------------


% Copy this if needed to customize:
\input{./bibs/latex_templates/editingmarks}

% If we are using Haskell code in this paper:

\usepackage{upquote}
\usepackage{listings}
\usepackage[usenames,dvipsnames]{xcolor}

\definecolor{darkgreen}{rgb}{0,0.5,0}
\definecolor{darkred}{rgb}{0.5,0,0}

% Define the language styles we will use
%
\lstset{%
    frame=none,
    rulecolor={\color[gray]{0.7}},
    numbers=none,
    basicstyle=\ttfamily,        
%    basicstyle=\smallsize\ttfamily,    
%    basicstyle=\footnotesize\ttfamily,
%    basicstyle=\scriptsize\ttfamily,
%    basicstyle=\Largesize\ttfamily,        
    numberstyle=\color{Gray}\tiny\it,
    commentstyle=\color{MidnightBlue}\it,
    stringstyle=\color{Maroon},
    keywordstyle=[1],
    keywordstyle=[2]\color{ForestGreen},
    keywordstyle=[3]\color{Bittersweet},
    keywordstyle=[4]\color{RoyalPurple},
    captionpos=b,
    aboveskip=1\medskipamount,
    xleftmargin=0.5\parindent,
    xrightmargin=0.5\parindent,
    flexiblecolumns=false,
%   basewidth={0.5em,0.45em},           % default {0.6,0.45}
%    escapechar={\%},
    escapechar={\@},
    mathescape=true,
    texcl=true                          % tex comment lines
}

\lstloadlanguages{Haskell}
\lstdefinestyle{haskell}{%
    language=Haskell,
    upquote=true,
    basicstyle=\ttfamily,        
    deletekeywords={case,class,data,default,deriving,do,in,instance,let,of,type,where,IO,ST,STM,read},
%    morekeywords={[1]read,write,finish},
    morekeywords={[2]class,data,default,deriving,family,instance,type,where},
    morekeywords={[3]in,let,case,of,do,switch},
    morekeywords={[4]IORef,IO,ST,STM,Symbol},
    literate=
        {\\}{{$\lambda$}}1
        {\\\\}{{\char`\\\char`\\}}1
        {>->}{>->}3
        {>>=}{>>=}3
        {->}{{$\rightarrow$}}2
        {>=}{{$\geq$}}2
        {<-}{{$\leftarrow$}}2
        {<=}{{$\leq$}}2
        {=>}{{$\Rightarrow$}}2
        {|}{{$\mid$}}1
        {forall}{{$\forall$}}1
        {exists}{{$\exists$}}1
        {...}{{$\cdots$}}3
%       {`}{{\`{}}}1
%       {\ .}{{$\circ$}}2
%       {\ .\ }{{$\circ$}}2
%
%    deletekeywords={insert},
%    deletekeywords={map,sort,zipWith,replicate,Num,Char,Bool,Array,Int,Double
%                   ,sqrt,not,filter,IO,Maybe,Either,quot,scanl,scanr,reverse,fst,id},
%    literate=
%        {+}{{$+$}}1
%        {/}{{$/$}}1
%        {*}{{$*$}}1
%        % {=}{{$=$}}1
%        {>}{{$>$}}1 {<}{{$<$}}1
%        {\\}{{$\lambda$}}1
%        {\\\\}{{\char`\\\char`\\}}1
%        {->}{{$\rightarrow\;$}}2
%        {>=}{{$\geq$}}2
%        {<-}{{$\leftarrow\;$}}2
%        {<=}{{$\leq$}}2
%        {=>}{{$\Rightarrow\;$}}2
%        {\ .}{{$\circ$}}2
%        {\ .\ }{{$\circ$}}2
%        {>>}{{>>}}2
%        {>>=}{{>>=}}2
%        {=<<}{{=<<}}2
%        {|}{{$\mid$}}1
%        {dotdotdot}{{$\ldots$}}3
}

\lstdefinestyle{inline}{%
    style=haskell,
%    basicstyle=\footnotesize\ttfamily,
    basicstyle=\ttfamily,
    %% keywordstyle=[1],
    %% keywordstyle=[2],
    %% keywordstyle=[3],
    %% keywordstyle=[4],
    literate=
        {\\}{{$\lambda$}}1
        {\\\\}{{\char`\\\char`\\}}1
        {>->}{>->}3
        {>>=}{>>=}3
        {->}{{$\rightarrow$\space}}3    % include forced space
        {>=}{{$\geq$}}2
        {<-}{{$\leftarrow$}}2
        {<=}{{$\leq$}}2
        {=>}{{$\Rightarrow$}}2
        {|}{{$\mid$}}1
%        {~}{{$\sim$}}1
        {forall}{{$\forall$}}1
        {exists}{{$\exists$}}1
        {...}{{$\cdots$}}3
}

\lstnewenvironment{code}
    {\lstset{style=haskell}%
      \csname lst@SetFirstLabel\endcsname}
    {\csname lst@SaveFirstLabel\endcsname}
    {}

% Default all listings to Haskell style
\lstset{style=haskell}

% \newcommand{\inl}[1]{\lstinline[style=inline];#1;}
\newcommand{\il}[1]{\lstinline[style=inline];#1;}

\newcommand{\makeatcode}{\lstMakeShortInline[style=inline]@}
\newcommand{\makeatchar}{\lstDeleteShortInline@}


% Sometimes we have extra short-cuts:
% \input{macro-defs}

% Sometimes we factor large figures into commands in a separate file:
% \input{figures}

\begin{document}

\special{papersize=8.5in,11in}
\setlength{\pdfpageheight}{\paperheight}
\setlength{\pdfpagewidth}{\paperwidth}

\conferenceinfo{CONF 'yy}{Month d--d, 20yy, City, ST, Country}
\copyrightyear{20yy}
\copyrightdata{978-1-nnnn-nnnn-n/yy/mm}
\copyrightdoi{nnnnnnn.nnnnnnn}

% Uncomment the publication rights you want to use.
%\publicationrights{transferred}
%\publicationrights{licensed}     % this is the default
%\publicationrights{author-pays}

\titlebanner{banner above paper title}        % These are ignored unless
\preprintfooter{short description of paper}   % 'preprint' option specified.

\title{Compiling tree transformations on packed representations}
\subtitle{Faster compiler passes, parallelization ready}

\authorinfo{Name1}
           {Affiliation1}
           {Email1}
\authorinfo{Name2\and Name3}
           {Affiliation2/3}
           {Email2/3}

\maketitle

\begin{abstract}
This is the text of the abstract.
\end{abstract}

\category{CR-number}{subcategory}{third-level}

% general terms are not compulsory anymore,
% you may leave them out
\terms
term1, term2

\keywords
keyword1, keyword2

% ================================================================================
\section{Introduction}
% ================================================================================

The text of the paper begins here.

\rn{This is an example peanut-gallery comment.}

\note{We use these bullets for outlining / psuedo-text.  They're removed when
  converted to actual prose.}


% ================================================================================
\section{Background}
% ================================================================================


% ================================================================================
\section{Formal language}
% ================================================================================

\note{The calculus, plus the core lowering transforms in figures.}

We present two languages: L1, a simple purely functional, first-order programming
language with algebraic data types; and L2, an imperative programming language
with mutable arrays and pointer arithmetic, as well as a type-and-effect system.

\subsection{L1: Source language}

\subsection{L2: Target language}

\begin{displaymath}
  \begin{aligned}
    v \in & Var \\
    n \in & Int
  \end{aligned}
\end{displaymath}
\begin{displaymath}
  \begin{aligned}
    \keywd{x} \gramdef & \keywd{n} \gramor \gramor{v} \\
    \keywd{e} \gramdef & \keywd{i} \gramor \keywd{prim} \keywd{x}* \\
    & \gramor \keywd{(x,x,...)} \gramor \keywd{x} @ \keywd{n} \\
    & \gramor \keywd{let} \keywd{(v,t,e)}* \keywd{e} \\
    & \gramor \keywd{ifeq} \keywd{(x,x)} \keywd{e} \keywd{e} \\
    & \gramor \keywd{newbuff} n \\
    & \gramor \keywd{writetag} \keywd{v} \keywd{n} \\
    & \gramor \keywd{writeint} \keywd{v} \keywd{x} \\
    & \gramor \keywd{readtag} \keywd{v} \\
    & \gramor \keywd{readint} \keywd{v}
  \end{aligned}
\end{displaymath}


% ================================================================================
\section{Implementation}
% ================================================================================



% ================================================================================
\section{Evaluation}
% ================================================================================



\subsection{Additional tree benchmarks}
% ----------------------------------------

\note{Non-compiler benchmark(s)}


% ================================================================================
\section{Future work}
% ================================================================================

\note{Data type factoring, storing leaves in a separate, dense, aligned vector.
This enables (1) vectorization of numeric operations, and (2) separating
out pointers that the GC must traverse.  This can prove essential for an
open-world implementation in a managed language.}



% ================================================================================
\appendix
\section{Appendix Title}
% ================================================================================

This is the text of the appendix, if you need one.

\acks

Acknowledgments, if needed.

\bibliographystyle{abbrvnat}

% If you can't commit to the submodule right this second, just copy
% this file to ./refs.bib :
\bibliography{bibs/refs}

% The bibliography should be embedded for final submission.
%% \begin{thebibliography}{}
%% \softraggedright
%% \bibitem[Smith et~al.(2009)Smith, Jones]{smith02}
%% P. Q. Smith, and X. Y. Jones. ...reference text...
%% \end{thebibliography}


\end{document}


